% Created 2024-11-30 Sat 23:21
% Intended LaTeX compiler: pdflatex
\documentclass{article}
\usepackage[utf8]{inputenc}
\usepackage[T1]{fontenc}
\usepackage{graphicx}
\usepackage{longtable}
\usepackage{wrapfig}
\usepackage{rotating}
\usepackage[normalem]{ulem}
\usepackage{amsmath}
\usepackage{amssymb}
\usepackage{capt-of}
\usepackage{hyperref}

\usepackage{fancyhdr}
\usepackage[top=25mm, left=25mm, right=25mm]{geometry}
\usepackage{longtable}
\fancyhead[R]{}
\setlength\headheight{43.0pt}
\author{Marco Marcillo}
\date{2024-11-30}
\title{Taller de Comandos Emacs}
\hypersetup{
 pdfauthor={Marco Marcillo},
 pdftitle={Taller de Comandos Emacs},
 pdfkeywords={},
 pdfsubject={},
 pdfcreator={Emacs 27.1 (Org mode 9.7.5)}, 
 pdflang={Español}}
\begin{document}

\maketitle
\fancyhead[C]{\includegraphics[scale=0.05]{./images/logoEPN.jpg}\\
ESCUELA POLITÉCNICA NACIONAL\\FACULTAD DE INGENIERÍA DE SISTEMAS\\
ARQUITECTURA DE COMPUTADORES}
\thispagestyle{fancy}






\section{Objetivos}
\label{sec:orgf910b6f}

\begin{itemize}
\item Comprender y utilizar comandos de Emacs para poder navegar de manera
\end{itemize}
eficiente por el texto y editarlo.

\begin{itemize}
\item Aprender a manipular archivos en Emacs
\end{itemize}

\section{Instrucciones}
\label{sec:orgdba02da}
Seguir el tutorial integrado en Emacs al respecto de la navegación y
operaciones más frecuentes. El tutorial puede ser accedido en Español
utilizando el comando:

\begin{verbatim}
M-x help-with-tutorial-spec-language
\end{verbatim}

Realice los ejercicios del tutorial (al menos un 80\% del texto) y
complete la siguiente tabla con los comandos que considere de mayor
interés. Verifique que en la parte superior se active el menú de
tabla. Dentro de la región de la tabla puede dar C-c C-c para que se
alinee automáticamente la tabla al contenido del texto que
escriba. Para generar una nueva fila escriba | y presione la tecla TAB

\section{Comandos Emacs}
\label{sec:orga01232a}
\begin{longtable}{llll}
\textbf{Comando} & \textbf{Descripción} & \textbf{Comando} & \textbf{Descripción}\\[0pt]
\hline
\endfirsthead
\multicolumn{4}{l}{Continued from previous page} \\[0pt]
\hline

\textbf{Comando} & \textbf{Descripción} & \textbf{Comando} & \textbf{Descripción} \\[0pt]

\hline
\endhead
\hline\multicolumn{4}{r}{Continued on next page} \\
\endfoot
\endlastfoot
\hline
\texttt{C-c C-e \# latex} & Insertar template de  latex & \texttt{C-x C-s} & Guardar los cambios en el archivo\\[0pt]
\texttt{C-x C-c \# latex} & Salir de emacs & \texttt{C-c C-z     \# latex} & Nos lleva a buffer interprete\\[0pt]
\texttt{M-w     \# latex} & Copiar & \texttt{C-w       \# latex} & Cortar\\[0pt]
\texttt{C-y     \# latex} & Pegar & \texttt{C-x u     \# latex} & Deshacer\\[0pt]
\texttt{C-x     \# latex} & Minimizar & \texttt{C-c C-,   \# latex} & Insertar blog\\[0pt]
\texttt{C-c C-c \# latex} & Ejecutar & \texttt{C-x C-s   \# latex} & Guardar\\[0pt]
\texttt{C-x 1   \# latex} & Quedarse en una hoja & \texttt{C-espacio \# latex} & Seleccionar\\[0pt]
\texttt{C-h     \# latex} & Ayuda & \texttt{C-c C-l   \# latex} & Insertar archivos\\[0pt]
\texttt{C-g     \# latex} & Cancelar un comando en curso & \texttt{M-g g     \# latex} & Ir a una linea especifica\\[0pt]
\texttt{C-v     \# latex} & Avanzar una pagina & \texttt{M-v       \# latex} & Retroceder una pagina\\[0pt]
\texttt{C-a     \# latex} & Ir al inicio de la linea & \texttt{C-e       \# latex} & Ir al final de una linea\\[0pt]
\texttt{M-/     \# latex} & Autocompletar palabras & \texttt{C-s       \# latex} & Busqueda hacia adelante\\[0pt]
\texttt{C-r     \# latex} & Busqueda hacia atras & \texttt{M-\%       \# latex} & Remplazar texto\\[0pt]
\texttt{C-x b   \# latex} & Cambiar entre buffers & \texttt{C-l       \# latex} & Centrar linea actual\\[0pt]
\texttt{C-x 2   \# latex} & Dividir la ventana en horizontal & \texttt{C-x 2     \# latex} & Dividir la pantalla en vertical\\[0pt]
\texttt{C-x o   \# latex} & Cambiar entre ventana & \texttt{C-x k     \# latex} & Cerrar el buffer actual\\[0pt]
\texttt{C-x C-b \# latex} & Mostar una lista de buffers & \texttt{C-l       \# latex} & Refresca la pantalla\\[0pt]
\texttt{C-n     \# latex} & salta una linea hacia abajo & \texttt{C-p       \# latex} & Sube una hacia la linea superior\\[0pt]
\end{longtable}
\end{document}

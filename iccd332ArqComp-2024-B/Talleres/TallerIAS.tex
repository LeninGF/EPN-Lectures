% Created 2024-12-23 Mon 19:53
% Intended LaTeX compiler: pdflatex
\documentclass{article}
\usepackage[utf8]{inputenc}
\usepackage[T1]{fontenc}
\usepackage{graphicx}
\usepackage{longtable}
\usepackage{wrapfig}
\usepackage{rotating}
\usepackage[normalem]{ulem}
\usepackage{amsmath}
\usepackage{amssymb}
\usepackage{capt-of}
\usepackage{hyperref}

\usepackage{fancyhdr}
\usepackage[top=25mm, left=25mm, right=25mm]{geometry}
\usepackage{longtable}
\fancyhead[R]{}
\setlength\headheight{43.0pt}
\usepackage[T1]{fontenc}
\usepackage[utf8]{inputenc}
\usepackage[spanish]{babel}
\usepackage[backend=biber, style=apa]{biblatex}
\addbibresource{/home/alejandro0122/EPN-Lectures/iccd332ArqComp-2024-B/FormatoTareas/bibliography.bib}
\usetheme{default}
\author{Marco Marcillo}
\date{\today}
\title{TallerIAS}
\hypersetup{
 pdfauthor={Marco Marcillo},
 pdftitle={TallerIAS},
 pdfkeywords={},
 pdfsubject={},
 pdfcreator={Emacs 27.1 (Org mode 9.3)}, 
 pdflang={Spanish}}
\begin{document}

\maketitle
\begin{frame}{Outline}
\tableofcontents
\end{frame}

\fancyhead[C]{\includegraphics[scale=0.05]{.././FormatoTareas/logoEPN.jpg}\\
ESCUELA POLITÉCNICA NACIONAL\\FACULTAD DE INGENIERÍA DE SISTEMAS\\
ARQUITECTURA DE COMPUTADORES}
\thispagestyle{fancy}


\section{Recursos}
\label{sec:org0e2ec63}
Para este taller consulte el libro \citetitle{stallings2006}
disponible en \href{https://epnecuador-my.sharepoint.com/:f:/g/personal/lenin\_falconi\_epn\_edu\_ec/EgjH2RoedD5NuswqpOt8ExsB\_DE052v9Rlrg0QpEtbimDg?e=WoqexR}{EPN-SHAREPOINT(Clic Me)} y use su WSL con Emacs para editar el archivo.
\subsection{Observaciones y Recomendaciones}
\label{sec:orgaba6b63}
\begin{itemize}
\item Tenga en cuenta la ruta a las imágenes y a la bibliografía. Se le
\end{itemize}
recomienda que realice una clonación o un fork del repositorio de la
clase a fin de que no tenga problemas generando el documento.
\begin{itemize}
\item Instale \LaTeX completo en el computador que generará el archivo pdf.
\end{itemize}
\section{Instrucciones}
\label{sec:org7b9c085}
Complete las solicitudes del taller y suba el archivo \texttt{.ORG} y \texttt{.PDF}
al aula virtual. Cuando trabaje en Emacs trate de utilizar las
combinaciones de teclas para navegar por el documento.

\section{Taller Máquina de Von Neuman}
\label{sec:org87386bf}

La máquina de Von Neuman realiza operaciones en un ciclo repetitivo de
captación y ejecución. Para esto, el computador IAS usa una memoria de
40 bits. Cuando la memoria se usa numéricamente, se representa el
número en complemento a 2, reservando el bit inicial para el
signo. Cuando la memoria se usa para el registro de instrucciones, se
divide en dos partes de 20 bits. Los 8 primeros bits de cada parte
corresponden al \emph{opcode} y los restantes 12 bits de cada parte a la
dirección (operando) desde la que se debe leer o a la que se debe
escribir.

\subsection{Conjunto de Instrucciones IAS}
\label{sec:orgc555b26}
Complete la Tabla 2.1 de instrucciones del computador IAS
\autocite[p.47]{stallings2006} usando la notación RTL\footnote{Register Transfer Language o Lenguaje de Transferencia de Registros}. Se suministra
la Tabla en código \LaTeX para tal efecto. Edite el archivo en Emacs
usando el modo principal ORG y cambie al modo \LaTeX para editar la
Tabla. El cambio de modo se realiza usando \texttt{M-x Latex-mode RET} y para
retornar al modo \texttt{ORG} haga \texttt{M-x org-mode RET}

\begin{itemize}
\item \texttt{M} equivale a \emph{Alt}
\item \texttt{x} es la tecla \emph{x}
\item \texttt{RET} es presionar \emph{enter}
\end{itemize}

Una vez activado el modo \LaTeX el siguiente código recibe colores
sobre las palabras clave.

\begin{table}[htbp]
  \caption{Instrucciones Maquina IAS}
  \label{tab:tablaISA-IAS}
  \centering
  \begin{tabular}{|cccc|}
    \hline
    Opcode   & Opcode Hex &  Simbolo          & RTL \\ \hline
    00001010 & 0xA        &  LOAD MQ          & $[AC]\leftarrow[MQ]$\\
    00001001 & 0x9        &  LOAD MQ, M(X)    & $[MQ]\leftarrow[X]$\\
    00100001 & 0x21       &  STOR M(X)        & $[X]\leftarrow[AC]$\\
    00000010 & 0x02       &  LOAD M(X)        & $[AC]\leftarrow[X]$\\
    00000011 & 0x03       &  LOAD |M(X)|      & $[AC]\leftarrow|[X]|$\\
    00000100 & 0x04       &  LOAD -|M(X)|     & $[AC]\leftarrow-|[X]|$\\
    00001101 & 0xD        &  JUMP M(X.0:19)   & $[PC]\leftarrow[X]$\\
    00001110 & 0xE        &  JUMP M(X.20:39)  & $[PC]\leftarrow[X]$\\
    00001111 & 0xF        &  JUMP M+(X.0:19)  & $Si [AC] >=0,  [PC]\leftarrow[X]$\\
    00010000 & 0x10       &  JUMP M+(X.20:39) & $Si [AC] >=0,  [PC]\leftarrow[X]$\\
    00000101 & 0x05       &  ADD M(X)         & $[AC]\leftarrow[AC]+[X]$\\
    00000110 & 0x06       &  SUB M(X)         & $[AC]\leftarrow[AC]-[X]$\\
    00000111 & 0x07       &  ADD IM(X)        & $[AC]\leftarrow[AC]+|[X]|$\\
    00001000 & 0x08       &  SUB IM(X)        & $[AC]\leftarrow[AC]-|[X]|$\\
    00001011 & 0xB        &  MUL M(X)         & $[AC]\leftarrow[AC]*[X]$\\
    00001100 & 0xC        &  DIV M(X)         & $[AC]\leftarrow[AC]/[X]$\\
    00010010 & 0x12       &  STOR M(X.8:19)   & $[X]\leftarrow[AC]$\\
    00010011 & 0x13       &  STOR M(X.28:39)  & $[X]\leftarrow[AC]$\\
    00010100 & 0x14       &  LSH              & $[AC]\leftarrow[AC]*2$\\
    00010101 & 0x15       &  RSH              & $[AC]\leftarrow[AC]/2$\\
    \hline
    
  \end{tabular}
\end{table}

\newpage

\subsection{Ejercicio}
\label{sec:org9a57614}

En la máquina IAS, las instrucciones se dividen en dos segmentos:
izquierdo desde el bit 0 a 19 y derecho desde el bit 20 al 39. Primero
se ejecuta el lado izquierdo (bits 0 a 19) y luego el derecho. El
contador de programa inicia en la posición 300. El set de
instrucciones del computador o ISA\footnote{Instruction Set Architecture: los códigos de programación} está definido en la Tabla
\ref{tab:tablaISA-IAS}. A fin de terminar el programa se agrega la
instrucción \texttt{0x99} para terminar el programa (i.e. HALT). Conteste las
siguientes preguntas considerando la Tabla \ref{tab:orgcd9552c} como el
mapa de Memoria de la máquina y obtenga el programa en Assembre y en
RTL.

si se tiene el siguiente mapa de memoria?

Si es un mapa de memoria ya que posee:

• Cada word es de 40 bits.

• Cada word se subdivide en 2 instrucciones del 20 bits
   en este caso del [0 al 19] y del [20 al 39].

• Tenemos la instruccion opcode HALT la cual finaliza el programa.

• Entre otras.

\begin{table}[htbp]
\caption{\label{tab:orgcd9552c}Mapa de Memoria}
\centering
\begin{tabular}{rrrrr}
\hline
Dirección & \(Opcode_1\) & \(X_1\) & \(Opcode_2\) & \(X_2\)\\[0pt]
\hline
0x300 & 0x01 & 0x940 & 0x06 & 0x941\\[0pt]
0x301 & 0x21 & 0x940 & 0x99 & 0x000\\[0pt]
\ldots{} & \ldots{} & \ldots{} & \ldots{} & \ldots{}\\[0pt]
0x940 & 0x00 & 0x000 & 0x00 & 0x005\\[0pt]
0x941 & 0x00 & 0x000 & 0x00 & 0x002\\[0pt]
\hline
\end{tabular}
\end{table}


\begin{frame}[label={sec:org8581f17}]{Solucion}
• Paso 1 (Opcode1)

LOAD M(940)
AC ← 0x003

\alert{PC} 0x300
\alert{AC} 0x003
\alert{IR} 0x940

• Paso 2 (Opcode2)

STORE M(941)
M(941) ← 0x003

\alert{PC} 0x300
\alert{AC} 0x003
\alert{IR} 0x941

• Paso 3 (Opcode1)

ADD M(941)
AC ← 0x003 + 0x003
AC ← 0x006

\alert{PC} 0x301
\alert{AC} 0x003
\alert{IR} 0x941

• Paso 4 (Opcode2)

HALT

\alert{Detiene la ejecución}
\end{frame}

\begin{frame}[label={sec:orgbd47b58}]{Preguntas}
\begin{enumerate}
\item ¿Qué resultado se tiene en el registro del acumulador?

El acumulador al final tiene la ejecución  0x006.

\item ¿Se sobrescribe algún registro como resultado de la ejecución?
Sí, se sobrescribe la dirección de memoria 0x941 con el valor 0x003
\end{enumerate}
\end{frame}

\begin{frame}[label={sec:org6305171}]{Código Assembler:}
\end{frame}

\section{Escriba el código Assembler del programa que ejecuta el computador IAS.}
\label{sec:orgbd47ef9}


\begin{verbatim}

• LOAD M(940)   -->  Cargar contenido de 0x940 en AC

• STORE M(941)  -->  Almacenar contenido de AC en 0x941

• ADD M(941)    -->  Sumar contenido de 0x941 a AC

• HALT          -->  Detener la ejecución)

\end{verbatim}

\begin{frame}[label={sec:orge4c0327}]{Código RTL}
Escriba en notación de transferencia de registros el programa que
ejecuta el computador


IAS \([AC] \leftarrow [940]\)

AC ← M[940]      -->  Cargar contenido de 0x940 en el AC(acumulador)

M[941] ← AC      -->  Almacenar contenido de AC en 0x941

AC ← AC + M[941] -->  Sumar contenido de 0x941 al acumulador

HALT(Ox99)       -->  Detener la ejecución


\printbibliography
\end{frame}
\end{document}
